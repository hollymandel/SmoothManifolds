\documentclass[10pt,letter]{article}
\usepackage{amsmath,amssymb,graphicx,setspace,fullpage,breqn,mathtools,stmaryrd}
\onehalfspacing
\usepackage{fullpage}
\DeclareMathOperator*{\argmin}{arg\,min}

\begin{document}
\noindent Github/dogestoyevsky \\
May 2017
\begin{center}
\textbf{Chapter One: Smooth Manifolds}\\ Lee, \textit{An Introduction to Smooth Manifolds}

\line(1,0){250}
\end{center}
\subparagraph{1-1} \textbf{The line with two origins.} Take $\mathbb{R} \times \lbrace 1 \rbrace \cup \mathbb{R} \times \lbrace -1 \rbrace \subset \mathbb{R}^2$ and let $L$ be the quotient described. Let $p$ be the quotient map.  

First, we show that $L$ is locally Euclidean. Cconsider the set $A_1 = \mathbb{R} \times \lbrace 1 \rbrace \subset L$. The preimage of this set under the quotient map can be written as the union
\begin{equation*}
\mathbb{R} \times \lbrace 1 \rbrace  \cup (-\infty,0) \times \lbrace -1 \rbrace \cup (0,\infty) \times \lbrace -1 \rbrace
\end{equation*}
which is open in the subspace topology on $\mathbb{R} \times \lbrace 1 \rbrace \cup \mathbb{R} \times \lbrace -1 \rbrace$. Now define $f: A_1 \rightarrow \mathbb{R}$ so that $f(x,1) = x$. Then $f$ is continuous: if $(a,b) \subset A_1$ is an interval not containing the origin, then $(f \circ p)^{-1}(a,b) = (a,b) \times \lbrace -1 \rbrace \cup (a,b) \times \lbrace 1 \rbrace$. Otherwise, $(f \circ p)^{-1}(a,b) = (a,b) \times \lbrace 1 \rbrace \cup (a,0) \times \lbrace -1 \rbrace \cup (0,b) \times \lbrace -1 \rbrace$. In addition, $f$ is bijective, since its inverse $g$ is given by $g(x) = (x,1)$. We claim that $g$ is continuous, so $f$ is a homeomorphism, and $(A_1,f)$ is a chart. (Then a similar argument for $A_2 = \mathbb{R} \times \lbrace -1 \rbrace \subset L$ gives a chart for the remaining points in $L$.)

To see that $f$ is an open mapping, we think about what subsets of $A_1$ can be open. Let $B$ be a saturated open subset of $\mathbb{R} \times \lbrace 1 \rbrace \cup \mathbb{R} \times \lbrace -1 \rbrace \subset \mathbb{R}^2$. It is easy to see that for non-origin point $z \in B$, there is an open neighborhood of $f \circ p(z)$ in $f \circ p(B)$. Now if $B$ contains $(0,1)$, then it must contain a neighborhood of $(-\epsilon,\epsilon) \times \lbrace 1 \rbrace$, in which case $f \circ p(B)$ contains $(-\epsilon,\epsilon)$, an open neighborhood of $f \circ p(0,1)$. If $B$ contains both origins, the situation is the same. If $B$ contains only $(0,-1)$, then $f \circ p(B)$ does not contain $0$. 


Secondly, similar to that in the previous paragraph shows that the quotient map $p$ is open. So $L$ is the open quotient of a second-countable space, and therefore second-countable.

Finally, $L$ is not Hausdorff. For any open subset of $L$ containing $(0,1)$ contains an open neighborhood $(-\epsilon,\epsilon) \times \lbrace 1 \rbrace$ whose preimage under $p$ contains $(-\epsilon,\epsilon) \times \lbrace 1 \rbrace \cup (-\epsilon,0) \times \lbrace -1 \rbrace \cup (0,\epsilon) \times \lbrace -1 \rbrace$. Similarly, the preimage under $p$ of any open subset of $L$ containing $(0,-1)$ contains $(-\epsilon,\epsilon) \times \lbrace -1 \rbrace \cup(-\epsilon',0) \times \lbrace 1 \rbrace \cup (0,\epsilon') \times \lbrace 1 \rbrace$. Since any two such sets overlap, it is impossible to find disjoint open sets separating $(0,1)$ and $(0,-1)$.

\subparagraph{1-2} \textbf{The disjoint union of uncountably many copies of $\mathbb{R}$.} Let $D = \coprod \limits_{\alpha \in \mathcal{A}} \mathbb{R} \times \lbrace \alpha \rbrace$ for an uncountable index set $\mathcal{A}$. $D$ is clearly locally Euclidean, since each $\mathbb{R} \times \lbrace \alpha \rbrace$ is itself a locally Euclidean neighborhood, and Hausdorff, since distinct points within each copy of $\mathbb{R}$ can be separated and points in different copies of $\mathbb{R}$ are separated by the corresponding copies of $\mathbb{R}$. However, any basis for the topology on $D$ must contain a subset of $(-\epsilon,\epsilon) \times \lbrace \alpha \rbrace$ for all $\alpha \in \mathcal{A}$. This collection alone is uncountable. 

\subparagraph{1-4} \textbf{Locally finite open covers.} \textbf{(a.)} If $p \in M$ is contained in infinitely many elements of $\mathcal{U}$, then any such element intersects infinitely many other elements of $\mathcal{U}$ at $p$, contradicting the assumption. \textbf{(b.)} Consider the topological manifold $T = \mathbb{R} \setminus 0$ with the open cover $\lbrace U_N = (-\infty,-N) \cup (N,\infty): N = 0,1,2,... \rbrace$. Any point $x \in T$ is only contained in finitely many $U_N$, since $x \in U_N$ exactly when $\vert x \vert > N$. However, all $U_N$ overlap. \textbf{(c.)}  Take $U \in \mathcal{U}$. By assumption, every $x \in \bar{U}$ is contained in an open neighborhood that intersects only finitely many elements of $\mathcal{U}$. The collection $\mathcal{U}'$ of these sets form an open cover of $\bar{U}$, so we can extract a finite subcover $U_1,...,U_N$. But every set $B \in \mathcal{U}$ intersecting $U$ must intersect one of the $U_k$, so there can only be finitely many such $B$.

\subparagraph{1-6} \textbf{A manifold with one smooth structure has infinitely many smooth structures.} For $s > 1$, $F_s(x) = \vert x \vert^{s-1}x$ is a homeomorphism $\mathbb{B}^n \rightarrow \mathbb{B}^n$. For it is bijective with inverse $G_s(x) = \vert x \vert^{\frac{1-s}{s}} x$, defining $G_s(0) = 0$. In addition, $F_s$ is clearly continuous, and $G_s$ is clearly continuous away from $0$. To see that $\lim \limits_{x \rightarrow 0} G_s(x) = 0$, we note that $\vert G_s(x) \vert = \vert x \vert^{\frac{1}{s}}$. However, $F_s$ is not a diffeomorphism for $s > 1$. For a computation shows that the Jacobian of $F_s$ at $0$ is the zero matrix, so $G_s$ cannot be differentiable at this point.

Now, let $A = \lbrace (U_{\alpha},\phi_{\alpha}): \alpha \in \mathcal{A} \rbrace$ be a smooth atlas on $M$ such that . We can assue that $\phi_{\alpha}(U_\alpha) = \mathbb{B}^n$ for all $\alpha \in \mathcal{A}$, for instance by taking each point $p \in M$ and restricting some chart containing $p$ to the preimage of a ball in $\mathbb{R}^n$, and then translating. We claim that for $s > 1$, $A_s = \lbrace (U_{\alpha},F_s \circ \phi_{\alpha}): \alpha \in \mathcal{A} \rbrace$ defines a smooth atlas on $M$. For each chart map is clearly still a homeomorphism of $U_{\alpha}$ with an open subset of $\mathbb{R}^n$, and for $x \in \phi_{\alpha}(U_{\alpha} \cap U_{\beta})$, 
\begin{equation*}
(F_s \circ \phi_{\beta}) \circ (F_s \circ \phi_{\alpha})^{-1} (x) = \phi_{\alpha}^{-1} \circ G_s \circ F_s \circ \phi_{\beta}(x) = \phi_{\alpha}^{-1}  \phi_{beta}(x),
\end{equation*}
so transition maps are smooth. However, for $s \neq t$,  $A_s \cup A_t$ is not a smooth atlas. THIS NOT TRUE. HOW TO FINISH?

\subparagraph{1-8} \textbf{Angle functions.} Say that $U$ is a proper subset of $S^1$. Assume without loss of generality that $U$ does not contain $(-1,0)$. It is a standard fact from complex analysis that there exists a holomorphic function $\text{log}: \mathbb{C} \setminus \mathbb{R}_{\leq 0} \rightarrow \mathbb{C}$ such that $e^{\text{log}(re^{i\theta})} = \text{log}(r) + i\theta$ for $\theta \in (-\pi,\pi)$. Restricting this function to $U$ and dividing by $i$ gives the desired map. 

On the other hand, there cannot be a continuous angle function  $\theta$  on $S^1$.  For say such a $\theta$ exists. Define a function $g: [-\pi,\pi] \rightarrow S^1$ by $g(x) = e^{ix}$ and consider the composition $\theta \circ g$ on $[-\pi,\pi]$. Clearly this function is continuous, and by the continuity of $\theta$, $\lim \limits_{x \rightarrow \pi} \theta \circ g(x) = \lim \limits_{x \rightarrow -\pi} \theta \circ g(x)$. Therefore by the intermediate value theorem there is a point $p \in (\pi,\pi)$ such that $\theta \circ g(p) = \theta \circ g(\pi)$. But since $g(p) \neq g(\pi)$, this proves that $\theta$ is not injective on $S^1$, which contradicts the definition of $\theta$.

Any angle function $\theta$ defined on a proper subset $U \subset S^1$ gives a chart for $S^1$ with the standard smooth structure. We demonstrate this on the chart $P: S^1 \cap \mathbb{H}^+ \rightarrow (-1,1)$ that sends $(x,y) \mapsto x$. First, consider $\theta \circ P^{-1}$ and take $t \in P(U \cap \mathbb{H}^+)$. Now since the derivative of $x \mapsto e^{ix}$ does not vanish on $\mathbb{C}$, it has a holomorphic inverse $f(x+iy) = f(x,y)$ in some neighborhood of $P(t)$. Then $\theta \circ P^{-1}(t) = f(t,\sqrt{1-t^2})$, which is smooth in $t$. On the other hand, for $t \in \theta(U \cap \mathbb{H}^+)$, $P \circ \theta^{-1}(t) = \cos(t)$, so this transition map is smooth as well. Similar arguments prove that the transition maps with the remaining charts of the standard smooth structure are smooth. By Proposition 1.17b, this proves that $(U, \theta)$ is a chart in the standard smooth structure. 

\subparagraph{1-10} \textbf{Smooth structure on the Grassmanian.} The matrix $B$ satisfies this condition. For given $v \in P$, the chosen basis representation allows us to write $\bigg( \begin{array}{c} I_k \\ B \end{array} \bigg)v = \bigg( \begin{array}{c} I_kv \\ Bv \end{array} \bigg) = v + Bv$. By the construction of $\phi$, the collection $\lbrace v + Bv: v \in P \rbrace$ spans $S$, so $B$ is as desired.

To show uniqueness, say $B'$ satsfies the same condition. Then the image of $P$ under the linear transformation $\bigg( \begin{array}{c} 0 \\ B-B' \end{array} \bigg)$ is contained in $S$. But by the choice of basis, this image is also contained in $Q$. Since $S \cap Q = \lbrace 0 \rbrace$, this proves that $B-B' = 0$. 

\subparagraph{1-12}. \textbf{Product of manifolds with a manifold with boundary.} Didn't write up. As far as I can tell it just goes through like you would expect. 





\end{document}